%-------------------------
% Resume in Latex
% Author : Jake Gutierrez
% Based off of: https://github.com/sb2nov/resume
% License : MIT
%------------------------

\documentclass[letterpaper,11pt]{article}

\usepackage{latexsym}
\usepackage[empty]{fullpage}
\usepackage{titlesec}
\usepackage{marvosym}
\usepackage[usenames,dvipsnames]{color}
\usepackage{verbatim}
\usepackage{enumitem}
\usepackage[hidelinks]{hyperref}
\usepackage{fancyhdr}
\usepackage[english]{babel}
\usepackage{tabularx}
\input{glyphtounicode}


%----------FONT OPTIONS----------
% sans-serif
% \usepackage[sfdefault]{FiraSans}
% \usepackage[sfdefault]{roboto}
% \usepackage[sfdefault]{noto-sans}
% \usepackage[default]{sourcesanspro}

% serif
% \usepackage{CormorantGaramond}
% \usepackage{charter}


\pagestyle{fancy}
\fancyhf{} % clear all header and footer fields
\fancyfoot{}
\renewcommand{\headrulewidth}{0pt}
\renewcommand{\footrulewidth}{0pt}

% Adjust margins
\addtolength{\oddsidemargin}{-0.5in}
\addtolength{\evensidemargin}{-0.5in}
\addtolength{\textwidth}{1in}
\addtolength{\topmargin}{-.5in}
\addtolength{\textheight}{1.0in}

\urlstyle{same}

\raggedbottom
\raggedright
\setlength{\tabcolsep}{0in}

% Sections formatting
\titleformat{\section}{
  \vspace{-4pt}\scshape\raggedright\large
}{}{0em}{}[\color{black}\titlerule \vspace{-5pt}]

% Ensure that generate pdf is machine readable/ATS parsable
\pdfgentounicode=1

%-------------------------
% Custom commands
\newcommand{\resumeItem}[1]{
  \item\small{
    {#1 \vspace{-2pt}}
  }
}

\newcommand{\resumeSubheading}[4]{
  \vspace{-2pt}\item
    \begin{tabular*}{0.97\textwidth}[t]{l@{\extracolsep{\fill}}r}
      \textbf{#1} & #2 \\
      \textit{\small#3} & \textit{\small #4} \\
    \end{tabular*}\vspace{-7pt}
}

\newcommand{\resumeSubSubheading}[2]{
    \item
    \begin{tabular*}{0.97\textwidth}{l@{\extracolsep{\fill}}r}
      \textit{\small#1} & \textit{\small #2} \\
    \end{tabular*}\vspace{-7pt}
}

\newcommand{\resumeProjectHeading}[2]{
    \item
    \begin{tabular*}{0.97\textwidth}{l@{\extracolsep{\fill}}r}
      \small#1 & #2 \\
    \end{tabular*}\vspace{-7pt}
}

\newcommand{\resumeSubItem}[1]{\resumeItem{#1}\vspace{-4pt}}

\renewcommand\labelitemii{$\vcenter{\hbox{\tiny$\bullet$}}$}

\newcommand{\resumeSubHeadingListStart}{\begin{itemize}[leftmargin=0.15in, label={}]}
\newcommand{\resumeSubHeadingListEnd}{\end{itemize}}
\newcommand{\resumeItemListStart}{\begin{itemize}}
\newcommand{\resumeItemListEnd}{\end{itemize}\vspace{-5pt}}

%-------------------------------------------
%%%%%%  RESUME STARTS HERE  %%%%%%%%%%%%%%%%%%%%%%%%%%%%


\begin{document}

%----------HEADING----------
% \begin{tabular*}{\textwidth}{l@{\extracolsep{\fill}}r}
%   \textbf{\href{http://sourabhbajaj.com/}{\Large Sourabh Bajaj}} & Email : \href{mailto:sourabh@sourabhbajaj.com}{sourabh@sourabhbajaj.com}\\
%   \href{http://sourabhbajaj.com/}{http://www.sourabhbajaj.com} & Mobile : +1-123-456-7890 \\
% \end{tabular*}

\begin{center}
    \textbf{\Huge \scshape TOUDGHI Ayoub} \\ \vspace{1pt}
    \small +33626176712 $|$ \href{mailto:x@x.com}{\underline{Toudghi.ayoub@gmail.com}} $|$ 
    \href{https://linkedin.com/in/...}{\underline{linkedin.com/in/ayoub-toudghi-477769134/}} $|$
    \href{https://github.com/...}{\underline{github.com/AybTGH}}
\end{center}


%-----------EDUCATION-----------
\section{Éducation}
  \resumeSubHeadingListStart
    \resumeSubheading
      {École Centrale de Lyon}{Lyon, France}
      {Programme Digital Lab \& 3\textsuperscript{ème} année en option informatique}{Sep. 2022 -- Sep. 2024}
    \resumeSubheading
      {École Centrale de Casablanca}{Casablanca, Maroc}
      {Ingénieur Généraliste}{Sep. 2020 -- Sep. 2024}
  \resumeSubHeadingListEnd




%-----------EXPERIENCE-----------
\section{Expériences Professionnelles}
  \resumeSubHeadingListStart

    \resumeSubheading
    {Stage en Data Science}{Avr. 2024 -- Sep. 2024}      
    {Dassault Systems }{velizy villacoublay, France}
      
    \resumeItemListStart
        \resumeItem{Amélioration de la précision des prévisions de séries temporelles avec le modèle Prophet, réduisant l'erreur MASE de 5 à 10 \% grâce à une approche bayésienne pour détecter les points de changement.}
        \resumeItem{Optimisation des hyperparamètres du modèle UnSupervised Anomaly Detection (USAD), augmentant la précision de détection des anomalies de 20 \% pour des séries temporelles multivariées.}
        \resumeItem{Réalisation d’un clustering avec KMeans basé sur la distance DTW (Dynamic Time Warping) pour regrouper des séries temporelles similaires, atteignant un score silhouette de 0,7.}
    \resumeItemListEnd


\resumeSubheading
    {Stage en Data Science}{Mar. 2023 -- Août 2023}    
    {Airbus}{Toulouse, France}
    \resumeItemListStart
        \resumeItem{ Extraction et standardisation de plus de 2 000 descriptions d’anomalies (snag job descriptions) à partir des carnets de bord des avions de la famille A320, améliorant la qualité des données de 25\% et facilitant l'analyse en aval.}
        \resumeItem{Application de techniques de clustering sur les snag job descriptions pour identifier les problèmes récurrents, augmentant le score silhouette de 0,3 à 0,8 (+167 \%), ce qui a permis une identification et une résolution plus rapide des problèmes de maintenance aéronautique.}
        \resumeItem{Industrialisation des algorithmes de clustering et de traitement de texte avec AWS et PySpark, réduisant les temps de traitement des données de 40 \% sur de grands volumes.}
    \resumeItemListEnd


% \resumeSubheading
%     {Data Scientist/Engineer Consultant}{Jan. 2023 -- Mar. 2023}
%     {BOSCH REXROTH}{Lyon, France}
%     \resumeItemListStart
%         \resumeItem{Developed a scalable SQL database integrated with a FastAPI backend and React front-end, optimizing data processing and enhancing user interaction for a seamless experience.}
%         \resumeItem{Applied advanced data science techniques, including classification models, to analyze and segment client data, enabling more informed and data-driven business decisions.}
%     \resumeItemListEnd

% \resumeSubheading
%     {Data Scientist Consultant}{Nov. 2022 -- Dec. 2022}
%     {TECHNIVUE}{Lyon, France}
%     \resumeItemListStart
%         \resumeItem{Created a Convolutional Neural Network (CNN)-based tool to process drone-captured images, identifying and counting plants and their substructures with advanced image processing techniques.}
%         \resumeItem{Achieved a 20\% error rate in plant counting, significantly advancing automated agricultural monitoring and supporting more efficient crop analysis.}
%     \resumeItemListEnd

      
% -----------Multiple Positions Heading-----------
%    \resumeSubSubheading
%     {Software Engineer I}{Oct 2014 - Sep 2016}
%     \resumeItemListStart
%        \resumeItem{Apache Beam}
%          {Apache Beam is a unified model for defining both batch and streaming data-parallel processing pipelines}
%     \resumeItemListEnd
%    \resumeSubHeadingListEnd
%-------------------------------------------



%-----------PROJECTS-----------
\section{Projets}
    \resumeSubHeadingListStart

    \resumeProjectHeading
      {\textbf{Consultant Data Science chez BOSCH REXROTH} $|$ \emph{React, FastAPI, SQL, Data Engineering}}\\
      {Jan. 2023 -- Mar. 2023}
    
    \resumeItemListStart
        \resumeItem{Développement d’une application web d’aide à la décision pour les commerciaux : création de tableaux de bord interactifs réduisant de 40 \% le temps de recherche des informations clients et accords déjà conclus.}
        \resumeItem{Mise en place d’une base de données SQL évolutive et de modèles de classification : amélioration de 25 \% de la précision de segmentation client, facilitant des décisions commerciales plus rapides et pertinentes.}
    \resumeItemListEnd





    \resumeProjectHeading
      {\textbf{Consultant Computer Vision / Software Engineer chez Technivue} $|$ \emph{OpenCV, Image Processing, Computer Vision}}\\
      {Nov. 2022 -- Dec. 2022}
    \resumeItemListStart
        \resumeItem{Développement d’un pipeline de traitement d’images aériennes de très grande taille (+1 Go) : découpage et prétraitement optimisés, réduisant de 50 \% le temps d’analyse.}
        \resumeItem{Mise en place d’un modèle CNN pour le comptage automatique des plantes : atteinte d’une précision de 80 \% et amélioration de la surveillance agricole.}
    \resumeItemListEnd


%
%-----------PROGRAMMING SKILLS-----------
\section{Compétences Techniques}
\resumeItemListStart
  \resumeItem{\textbf{Programming:} Python, Java, C++}
  \resumeItem{\textbf{Data Science \& AI:} Machine Learning, Deep Learning, Natural Language Processing (NLP), Python (PyTorch, TensorFlow, Keras), Large Language Models (RAG, LangChain)}
    \resumeItem{\textbf{Data Engineering:} Git, Apache Spark, AWS (SageMaker, S3)}
  \resumeItem{\textbf{Database Management:} SQL, PostgreSQL, SparkQL, MongoDB}
\resumeItemListEnd


\section{Certifications}
\resumeItemListStart
    \resumeItem{\textbf{Deep Learning Specialization} }
    \resumeItem{\textbf{Generative AI with Large Language Models}}

\resumeItemListEnd

%-------------------------------------------
\end{document}
